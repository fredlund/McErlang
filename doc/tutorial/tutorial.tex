\documentclass[a4paper]{article} 
%%\documentclass{llncs} 

\usepackage{a4wide}
\usepackage{times}
\usepackage{xspace}
\usepackage{epsfig}
\usepackage{listings}
\usepackage{color}
\usepackage{hyperref}

\newcommand{\erlang}{Erlang\xspace}
\newcommand{\erlangOTP}{Erlang/OTP\xspace}
\newcommand{\mcerlang}{McErlang\xspace}

\lstdefinelanguage{Erlang}%
  {morekeywords={abs,after,and,apply,atom,atom_to_list,band,binary,%
      fun,binary_to_list,binary_to_term,bor,bsl,bsr,bxor,case,catch,%
      date,div,element,erase,end,exit,export,float,float_to_list,%
      get,halt,hash,hd,if,info,import,integer,integer_to_list,%
      length,link,list,list_to_atom,list_to_float,list_to_integer,%
      list_to_tuple,module,node,nodes,now,of,or,pid,port,ports,%
      put,receive,reference,register,registered,rem,%
      round,self,setelement,size,spawn,throw,tl,trace,trunc,%
      spawn_link,process_flag,
      tuple,tuple_to_list,unlink,unregister,whereis,try,%
      infinity,undefined,when},%
   otherkeywords={->,!,[,],\{,\}},%
   morecomment=[l]\%,%
   morestring=[b]",%
   morestring=[b]'%
  }

\lstset{language=Erlang}
%%\lstset{basicstyle=\small}
%%\lstset{keywordstyle=\small\bfseries}
\lstset{keywordstyle=\bfseries}


%% algorithms
\newcommand{\algsimulation}{mce\_alg\_simulation\xspace}
\newcommand{\algsafetyclassic}{mce\_alg\_safety\xspace}

%% src
\newcommand{\compilemod}{mce\_erl\_compile\xspace}

%% scripts
\newcommand{\mcerl}{mcerl\xspace}

%% properties
\newcommand{\logon}{monSendLogon\xspace}

%% monitors
\newcommand{\statechange}{stateChange\xspace}
\newcommand{\monTest}{mce\_mon\_test\xspace}

%% examples
\newcommand{\simplemessengerdir}{McErlang/examples/Simple\_messenger\xspace}
\newcommand{\run}{run.erl\xspace}


\begin{document}
\title{McErlang: a Tutorial
\thanks{This work has been partially supported by the FP7-ICT-2007-1 
Objective 1.2. IST number 215868}}

\author{
Clara Benac Earle and 
 Lars-{\AA}ke Fredlund\\
Babel group, LSIIS, Facultad de Inform\'atica, \\
Universidad Polit\'ecnica de Madrid, Spain\\
{\small\texttt{email: \{cbenac,lfredlund\}@fi.upm.es}}}


\maketitle 
\date
\section{Prerequisites}

We assume that you will be running McErlang under the Linux operating
system; we have been using Ubuntu 8.10 to develop and test the
tool. Moreover we assume that Erlang is already installed on your
computer and that the command erl for starting Erlang can be run
directly without specifying a directory path. The model checker was
developed under Erlang/OTP R12B-5; it will probably work under other
releases of Erlang/OTP as well but we haven't verified this.

We recommend that you have the User Manual at hand to help you
understanding this tutorial.

\section{Installing McErlang}

For the development of McErlang source code, documentation and examples we
are using the Subversion revision control system. This means that there is a
central repository where the code is located. To get your own local
copy of the repository type the 
following command in a terminal window.

\begin{verbatim}
$ svn checkout https://babel.ls.fi.upm.es/svn/McErlang/trunk McErlang
\end{verbatim}

This command will create a local copy of all the McErlang code and
examples. To compile McErlang it normally suffices to execute
``./configure; make'' in the main McErlang directory.
In the following we assume that McErlang is installed in the
directory \verb@~/McErlang@.

\begin{verbatim}
$ cd McErlang; ./configure; make
\end{verbatim}

The following examples assume that you have 
put the directory \verb@McErlang/scripts@
in the ``command path'' of your shell.
Consult the manual for your command shell if you are not sure how to do this.
As an example, 
if you are running a bash shell, the following piece of text
can be added to the file \verb@.profile@:
\begin{verbatim}
# set PATH so it includes the directory with McErlang scripts
if [ -d ~/McErlang/scripts ] ; then
    PATH=~/McErlang/scripts:"${PATH}"
fi
\end{verbatim}

\section{The Messenger example}

To illustrate the use of McErlang we consider the messenger example
from the ``Getting started with Erlang'' document in the Erlang/OTP
R12B documentation\footnote{Section 3.5 in
  \texttt{http://erlang.org/doc/getting\_started/part\_frame.html}.}.
To begin working with the example, locate the source code for the
messenger example ({\bf messenger.erl}) in the \simplemessengerdir
directory.

The messenger is a program which allows users to log in on different
nodes and send simple messages to each other. The messenger allows
"clients" to connect to a central server, specifying their identities
and locations that are then stored in the state of the server. Thus a
user won't need to know the name of the Erlang node where a user is
located to send a message to that user, only his identity.

Let us consider the example with three nodes depicted in
Fig.\ \ref{logon}, the node server\_node where the messenger server is
already running and two other nodes, Node1 and Node2. If a process in
Node1 calls the messenger:logon(clara) command the result of this call
is the spawning of a process registered as mess\_client that will send
a message to the messenger server containing the name of the user and
the pid of the mess\_client. The messenger will then look into its
state and if it is empty it will add the new user to the UsersList and
send a confirmation message to the mess\_cient.
 

\begin{figure}[Htbp]
\begin{center}
   \includegraphics*[width=0.75\textwidth]{logon}
   \caption{User clara sends a logon message to the messenger server}
   \label{logon}
\end{center}
\end{figure}

Lets now assume that both a user clara and a user fred in
fig.\ \ref{send} are logged on and clara wants to send a message to
fred. Again the message will be sent from the mess\_client running in
Node1 to the messenger server which, after checking that both clara
and fred are logged on, will forward the message to the mess\_client
running in Node1.

\begin{figure}[Htbp]
\begin{center}
   \includegraphics*[width=0.75\textwidth]{send}
   \caption{User clara sends a message to user fred}
   \label{send}
\end{center}
\end{figure}

Upon a logoff, the mess\_client will send a message to the messenger
server that will produce the deletion of the user that wanted to log
off from the UsersList maintained by the messenger server.

\section{Executing the messenger example inside McErlang}

To run the messenger example inside McErlang we need to implement the
start up of the system, and simulate the actions of the (simulated)
users of the messenger service. This test-case for using the messenger
service is what we call a scenario. Random scenarios
can, for instance, be generated by using the Quickcheck 
tool (this possibility is not in the scope of this tutorial).

\subsection{Writing a scenario}

In the file {\bf scenario.erl}, in \simplemessengerdir directory, we
have defined a function named ``start'' which should be invoked with a
list of commands representing the commands that each user sends to the
messenger. The start function first spawns a process executing the
module:start\_server function on the node named ``server''; the result
is that the messenger process (the server) is created. Then a process
is created for each simulated user, running on separate nodes.  These
user processes send commands to the messenger process, which forwards
them to users on other nodes (in case the command is a request to send
a message to another user).

\begin{verbatim}
-module(scenario).
-export([start/1,start_clients/2,execute_commands/1]).

start(Commands) ->
    spawn(server_node,messenger,start_server,[]),
    start_clients(Commands,1).

start_clients([],N) -> ok;
start_clients([Commands|Rest],N) ->
    Node = list_to_atom("n"++integer_to_list(N)),
    spawn(Node,?MODULE,execute_commands,[Commands]),
    start_clients(Rest,N+1).

execute_commands(Commands) ->
    lists:foreach
    (fun (Command) ->
         case Command of
             {logon, Client} ->
                 messenger:logon(Client);
             {message,Receiver,Msg} ->
                 messenger:message(Receiver,Msg);
             logoff ->
                 messenger:logoff()
         end
     end, Commands).
\end{verbatim}

\subsection{Compiling the Messenger Example using McErlang}

To execute Erlang code under the McErlang model checker a translation
step is necessary, whereby Erlang code is translated into modified
Erlang code suitable for running under the control of the McErlang
model checker (which has its proper runtime system for managing
processes, communication and nodes).  Thereafter, a normal Erlang
compilation translates the modified source code into beam (or native)
object files.


To compile the Messenger example
type ``make'' in the \simplemessengerdir directory. 

\begin{verbatim}
~/McErlang/examples/Simple_messenger$ make
../../scripts/mcerl_compiler -sources messenger.erl monSendLogon.erl \
                             scenario.erl test.erl testl.erl tests.erl
\end{verbatim}


The code resulting from the source-to-source translation and the beam
files are stored in the ebin directory. In the ebin directory, if
we type ``\mcerl'', an erlang shell is started.

\begin{verbatim}
~/McErlang/examples/Simple_messenger$ cd ebin/
~/McErlang/examples/Simple_messenger/ebin$ ../../../scripts/mcerl
Erlang (BEAM) emulator version 5.6.5 [source] [smp:4] [async-threads:0] [hipe] [kernel-poll:false]

Eshell V5.6.5  (abort with ^G)
1> 
\end{verbatim}


From this shell we can invoke the verification functions implemented
in the {\bf \run} file in the \simplemessengerdir directory. We see some
examples in the following sections.

\section{Debugging the messenger example}

We can use the McErlang debugger to explore the execution of the
messenger example step by step. The debug() function in the {\bf \run}
module is used to start a debugging session. To debug the messenger
example we only need to specify the mce\_alg\_debugger algorithm, and
the program to debug, here the scenario module with a list containing
the parameters used by the start function. The list contains the
commands sent by the user clara (logon, send the message ``hola'' to
fred and logoff) and the commands sent by the user fred (logon).


\begin{verbatim}
debug() ->
  mce:start
    (#mce_opts
     {program={scenario,start,
               [[[{logon,clara},{message,fred,"hola"},logoff],
                [{logon,fred}]]]},
      algorithm={mce_alg_debugger,void}}).
\end{verbatim}

To start the debugging session we type run:debug() in the erlang shell. 

\begin{verbatim}
McErlang/src/McErlang/examples/Simple_messenger/ebin$ ../../../scripts/mcerl 
Erlang (BEAM) emulator version 5.6.4 [source] [smp:2] [async-threads:0] [hipe] [kernel-poll:false]

Eshell V5.6.4  (abort with ^G)
1> run:debug().
Starting McErlang model checker environment version 1.0 ...

Starting mce_alg_debugger(void) algorithm on program
scenario:start([[{logon,clara},{message,fred,"hola"},logoff],[{logon,fred}]])
with monitor mce_mon_test(ok)

...

At stack frame 0: transitions:

1: process <node0@esther,1>:
    run function scenario:start([[[{logon,clara},{message,fred,"hola"},logoff],
                          [{logon,fred}]]])


stack(@0)> 
\end{verbatim}


Note that for readability some printouts have been omitted in the example.

The debugger shows all the possible transitions from the current state
(here the initial state) to the next state. In this example, the only
possible transition is to execute the scenario:start function
in the process running on the node0@esther node. Thus, we
choose ``1.''.

\begin{verbatim}
stack(@0)> 1.

At stack frame 1: transitions:

1: node server_node:
    receive signal {spawn,{messenger,start_server,[]},no,{pid,node0@esther,1}}
   from node node0@esther
\end{verbatim}

Now the only possible transition is for the server\_node node to receive the message spawn from the process with pid 1 located in node0@esther.

\begin{verbatim}
stack(@1)> 1.

At stack frame 2: transitions:

1: process <server_node,2>:
    run function messenger:start_server([])

2: node node0@esther:
    receive signal {message,{pid,node0@esther,1},
                            {hasSpawned,{pid,server_node,2}}}
   from node server_node
\end{verbatim}

Now there are two possible transitions: to execute the
messenger:start\_server function or to receive a message at the node
node0@esther containing the pid of the process that has been spawned.

\section{Model Checking the Messenger example}


McErlang is a model checker for programs written in Erlang. The idea
is to replace the part of the standard Erlang runtime system that
concerns distribution, concurrency and communication with a new
runtime system which simulates processes inside the model checker, and
which offers easy access to the program state. 

A transition system is a digraph that represents the behaviour of a
system. The labeled transition system generated by the model checker
comprises two elements:

\begin{itemize}

\item system states which record the state of all nodes,
  all processes, all message queues, etc of the program to model check. 
  Such states are {\em stable} in the sense that all processes
  in a state are either waiting in a receive statement
  to receive a message, or have just been spawned.

\item transitions or computation steps 
  between a source state and a destination state. A transition is labeled by a
  sequence of actions that represent selecting one process
  in the source state which is ready to run, and letting it execute until 
  it is again waiting in a receive statement. The actions that label the
  transition are the side effects caused by the execution of the process
  (i.e., sending a message to another process, linking to another process,
  \ldots).
\end{itemize}
Thus the model checker uses an interleaving semantics to execute Erlang 
programs.

McErlang provides a number of different ways to formulate correctness
properties for checking on a given program. Correctness properties are
given in the form of a monitor/automaton that is executed in parallel
with the program, checking for errors along the execution
path. Monitors are either safety monitors or b\"{u}chi monitors which
encode linear temporal logic properties (LTL). Given a program and
such an automaton, McErlang will run them in lockstep letting the
automaton investigate each new program state generated.  If the
property does not hold, a counterexample (an execution trace) is
generated.


We show two examples of using safety
monitor in the following section and in
sect.\ ~\ref{implementingmonitor}. In sect.\ ~\ref{buchi} we show
examples of using b\"{u}chi monitors.


\subsection{Model Checking a simple property}

We can use the \algsafetyclassic algorithm to check that a property
holds on all the states of a program. We define the function
safety in the run.erl file.

%\begin{figure}
%\begin{lstlisting}
\begin{verbatim}
safety() ->
  mce:start
    (#ev_opts
     {program={scenario,start,
               [[[{logon,clara},{message,fred,"hola"},logoff],
                [{logon,fred}]]]},
      algorithm={mce_alg_safety,void}}).
\end{verbatim}
%\end{lstlisting}
%\caption{safety classic}
%\label{fig:safety_classic}
%\end{figure}


Because we do not specify a monitor, the default monitor \monTest is
used, which is always true. However, the model checker will check that
no process terminates abnormally.

Let us execute the safety function.

\begin{verbatim}

Eshell V5.6.4  (abort with ^G)
1> run:safety().
Starting McErlang model checker environment...
Starting mce_alg_safety(void) algorithm on program
 scenario:start([[{logon,clara},{message,fred,"hola"},logoff],[{logon,fred}]])
with monitor mce_mon_test(ok)

*** Run ending. 2036 states explored, stored states 836.


Execution terminated normally
Access result using mce:result()
ok
\end{verbatim}
Apparently no process terminated abnormally.

\section{Model checking using B\"{u}chi monitors}
\label{buchi}

The alternative we will consider here is to express the desired
property in Lineal Temporal Logic (LTL), and to use the LTL2Buchi tool
developed by Hans Svensson and distributed with McErlang to
automatically translate an LTL formulae into a B\"uchi monitor (see
User Manual).

We have model checked several interesting safety and liveness
properties of the messenger example. Some of them can be found in the
run.erl file. One such property can be informally expressed as
follows: if a user who is logged on sends a message to another user
who is also logged on then the recipient of the message will
eventually receive the message. As mentioned before, McErlang allows
access to the program states and the actions between states. We
explain in the following sections how this information can be used to
write properties.

\subsection{Using probe actions to write properties}

One possible and more precise description of
the aforesaid property is the following: 

\begin{quote}
\em if user1 does not send a
message m to user2 until user2 is logged on,
then if user1 does send a
message m to user2 then eventually user2 receives the message m.
\end{quote}

The formalization in LTL of the above formula can be found in the
``message\_received1()'' function in the {\bf run.erl} file. 

\begin{verbatim}
"not P until Q => (eventually P => eventually R)"
\end{verbatim}

The predicates P, Q and R have the following meaning in the concrete
scenario where a user clara that first sends a logon message to the
messenger server, then sends the message ``hi'' to the user fred and
finally sends a logoff message, and a user fred that sends a logon
message and a logoff message:
\begin{itemize}
  \item $\mathit{P}$: clara sends message ``hi'' to fred
  \item $\mathit{Q}$: fred is logged on
  \item $\mathit{R}$: fred receives the message ``hi'' from clara
\end{itemize}

Linear Temporal Logic is defined over program runs:
$\mathit{P}\;\mathit{until}\;\mathit{Q}$ holds for a program run
if at every state of the run 
the $\mathit{P}$ predicate holds, until a state in the program run
is encountered where the $\mathit{Q}$ predicate holds (and $\mathit{Q}$
must hold for some state on the run).
$\mathit{eventually}\;\mathit{R}$ holds for a program run if the
predicate $\mathit{R}$ holds at some program state in the run.
Normal logical implication is denoted by the ``$\Rightarrow$'' symbol.


For simplicity and modularity the property and the predicates present
in the property are considered separately. To write the predicates or
basic facts in the formula (user1 sends a message m to user2, etc.)
McErlang allows access to the program states and the sequence of
actions labelling transitions between states.  These predicates can be
written directly in Erlang, using all the expressiveness of the
language. For example, a predicate stating that a user is logged on
can be implemented as a function ``logon'' that returns true when an
action corresponding to the user being logged on is found labelling a
transition.  The process of searching for the desired action is
simplified if we annotate the program with what we call ``probe
actions'', which serve to make the internal state of a program visible
to the model checker in a simple fashion.

In the messenger example we have annotated the program code with probe
actions that are referred to in the predicates. For example, the
following probe action has been added in messenger.erl to the client
function for expressing that a user is logged on:
\lstset{emph={mce_erl,probe,probe_state,has_probe_state,del_probe_state},emphstyle=\underbar}
\begin{lstlisting}
client(Server_Node, Name) -> 
  {messenger,Server_Node}!{self(),logon,Name}, 
  await_result(), 
  mce_erl:probe(logon,Name),
  client(Server_Node).
\end{lstlisting}

From the example we can see that a probe action,
as created using the \lstinline@mce_erl:probe@ function, has two arguments, 
corresponding to a label naming the particular probe, 
and an arbitrary Erlang term as probe argument.

The code implementing predicates ({\bf basicPredicates.erl}) is in the
\simplemessengerdir directory. For example, the ``logon'' predicate is
implemented as the ``logon'' function which provided a user name as
argument, defines an anonymous function that returns true if its
second argument is a sequence of actions containing a logon probe
action corresponding to a logon by the named user:
\begin{lstlisting}
logon(Name) ->
  fun (_,Actions,_) ->
    lists:any
    (fun (Action) ->
       try
	 logon = 
           mce_erl_actions:
             get_probe_label(Action),
	 Name = 
           mce_erl_actions:
             get_probe_term(Action),
	 true
       catch _:_ -> false end
     end, Actions)
  end.
\end{lstlisting}

Similarly, probe actions and predicates have been written for the other
predicates appearing in property (1).

To model check property (1) on a concrete scenario we use the
function message\_received in run.erl. 



\begin{verbatim}
Eshell V5.6.5  (abort with ^G)
1> run:message_received1().
Starting McErlang model checker environment version 1.0 ...

Starting mce_alg_buechi(void) algorithm on program
scenario:start([[{logon,clara},{message,fred,"hi"},logoff],[{logon,fred},logoff]])
with monitor messenger_mon({void,[{'P',#Fun<basicPredicates.5.8045620>},
                                  {'Q',#Fun<basicPredicates.1.30432014>},
                                  {'R',#Fun<basicPredicates.6.70178892>}]})

...

Access result using mce:result()
To see the counterexample type "mce_erl_debugger:start(mce:result()). "
ok
2> 
\end{verbatim}

If we look at the counterexample we should be able to see that the
property did not hold because fred could logoff before receiving the
message.

One option to address the problem found is to generate only
test cases where fred never logs out. However, we prefer to
instead rewrite the property to
handle the situation when fred logs out. 

Thus we modify the property (1) as follows:
\begin{quote}
\em 
if user1 does not send a message m to user2 until user2 is logged on,
then if user1 does send a message m to user2 then eventually user2
receives the message m from user1, or user2 is logged off.
\end{quote}
The resulting LTL formula is the following:

\begin{verbatim}
"not P until Q => (eventually P => eventually (R or S))"
\end{verbatim}


where S represents the predicate ``fred is logged off''. This property
is used in the message\_received2 function in the {\bf run.erl} file with
the same scenario.

\begin{verbatim}
Eshell V5.6.5  (abort with ^G)
1> run:message_received2().
Starting McErlang model checker environment version 1.0 ...

Starting mce_alg_buechi(void) algorithm on program
scenario:start([[{logon,clara},{message,fred,"hi"},logoff],[{logon,fred},logoff]])
with monitor messenger_mon({void,[{'P',#Fun<basicPredicates.5.8045620>},
                                  {'Q',#Fun<basicPredicates.1.30432014>},
                                  {'R',#Fun<basicPredicates.6.70178892>},
                                  {'S',#Fun<basicPredicates.2.72475960>}]})

...

*** Run ending. 1004 states explored, stored states 1922.

Execution terminated normally
Access result using mce:result()
ok
2> 

\end{verbatim}

The property (2) has been checked against several scenarios, returning
always a positive result.

%As a side comment, maybe the result using this second property is not
%very satisfactory because we cannot prove that a message will be
%always eventually received. A naive way of checking that will be to
%eliminate from the scenario the logoff message sent by fred. A more
%serious approach is to write predicates that inspect the state instead
%of the actions. By doing that we could, for example, check that in
%every state we are interested in, fred occurs in the UserList
%maintained by the messenger server.

\subsection{Using probe states to write properties}

Working with probe actions in LTL formulas can sometimes be difficult,
as we have manually ``remember'' the occurrence of
important actions in the formula. 
In formulas (1) and (2) above, this was accomplished using the 
until formula.

Instead of using probe actions we can use so called ``probe states''.
In contrast to probe actions, which are enabled in a single
transition step only, such probes are persistent from the point in 
the execution of the program when they are enabled, until they 
are explicitly deleted.

As an example we instrument the login code of the server in
messenger.erl to record, using the function
\lstinline{mce_erl:probe_state}, the fact that a user has logged in:
\begin{lstlisting}
%%% Server adds a new user to the user list
server_logon(From, Name, User_List) ->
  %% check if logged on anywhere else
  case lists:keymember(Name, 2, User_List) of
    true ->
      From!{messenger,stop,
            user_exists_at_other_node},
      User_List;
    false ->
      From!{messenger, logged_on},
      mce_erl:probe_state({logged_on,Name}),
      [{From, Name}|User_List]
    end.
\end{lstlisting}

Similarly we delete the probe state using the function
\lstinline{mce_erl:del_probe_state} when a user logs out:
\begin{lstlisting}
%%% Server deletes a user from the user list
server_logoff(From,User_List) ->
  case lists:keysearch(From,1,User_List) of
    {value,{From,Name}} -> 
      mce_erl:del_probe_state
        ({logged_on,Name});
    false -> 
      ok
  end,
  lists:keydelete(From,1,User_List).
\end{lstlisting}

We can test for the existence of a probe state using the function
\lstinline@mce_erl:has_probe_state@ as exemplified in the function
\lstinline@logged_on@ in the basicPredicates.erl file which checks if
a user is logged on:
\begin{lstlisting}
logged_on(Name) ->
  fun (State,_,_) -> 
    mce_erl:has_probe_state
      ({logged_on,Name},State) 
  end.
\end{lstlisting}
This predicate will be abbreviated as ``t'' below.

We can now reformulate the second property above, removing
the until operator and one eventually operator with the
$\mathit{always}$ operator, which holds if its argument holds over
the whole execution trace:

\begin{verbatim}
"always ((P and T) => eventually (R or not T))"
\end{verbatim}


Note that since $\mathit{T}$ is a state predicate we can safely negate
it to compute its logical negation (``the user is not logged on'')
whereas the negation of the action predicate $\mathit{not Q}$ in
properties (1) and (2) only expresses that the ``logon action is not
present in the current transition'' (but it may have occurred earlier
in the execution of the program).

\subsection{Verification methodology}

A schema of the verification methodology used in this example is shown
in Fig.\ \ref{methodology}. 
\begin{figure}[Htbp]
\begin{center}
   \includegraphics*[width=0.5\textwidth]{methodology}
   \caption{verification schema}
   \label{methodology}
\end{center}
\end{figure}
To summarize, the steps we have followed to check that a program
verifies a property are the following:

\begin{itemize}
  \item Express the property in a suitable formalism, in this case, an
    LTL formula with some predicates. This is done by hand.
  \item Use the LTL2Buchi tool to translate the LTL property to a
    buchi automaton implemented as a monitor
  \item Annotate, by hand, the program with probe actions and/or probe 
states\footnote{Strictly speaking it is not necessary to annotate a program
using probe actions or probe states. McErlang can in principle
compute the information directly from inspecting the system state. However,
this is often not convenient as there is no way to gain access to
program data using variable names (variable names are not preserved
by the compilation process).}
  \item Invoke the McErlang model checker with the annotated program,
    the predicates, and the monitor.
\end{itemize}

The result provided by the model checker will be either a positive
result if the property holds or a counterexample (an execution trace)
if the property does not hold.



\section{Model Checking a Simple Safety Property}

\label{implementingmonitor}


An alternative approach to formulating the property in LTL and using
the LTL2B\"{u}chi translator is to implement a monitor/automaton
directly

Let us check a simple property: that a user never logs on. We
formulate this property as a safety property, ``something bad never
happens'', in this case the ``bad'' thing is if a user logs on. The
property is encoded as the \logon monitor given below. But
first we modify the client code to emit a synthetic
``probe'' action after having sent off a login message to the server:
\begin{verbatim}
client(Server_Node, Name) ->
    {messenger, Server_Node} ! {self(), logon, Name},
    mce_erl:probe(logon),
    await_result(),
    client(Server_Node).
\end{verbatim}
To implement a correctness property we implement a new module \logon
that provides a \statechange function, which is
called by McErlang whenever a new program state is encountered. 
When a logon action is found, the \statechange function returns a tuple, 
instead of returning a new monitor state, thus indicating an error to the
model checker. Note that instead of using a synthetic probe action we could
have simply checked for the existence of a send action with a logon
message inside.

\begin{verbatim}
-module(monSendLogon).
-export([init/1,stateChange/3,monitorType/0]).
-include("stackEntry.hrl").

-behaviour(mce_behav_monitor).

monitorType() ->
  safety.

init(State) ->
  {ok,State}.

stateChange(_,MonState,Stack) ->
  case has_probe_action(actions(Stack)) of
    {true, logon} -> {somebody,logon};
    no -> {ok,MonState}
  end.

actions(Stack) ->
  {Entry,_} = mce_behav_stackOps:pop(Stack),
  Entry#stackEntry.actions.
 
has_probe_action(Actions) ->
   mce_utils:findret(fun mce_erl:match_probe_label/1, Actions).
\end{verbatim}

To invoke the model checker with this property we have
defined the following function.

\begin{verbatim}
logon() ->
  mce:start
    (#ev_opts
     {program={scenario,start,
               [[[{logon,clara},{message,fred,"hola"},logoff],
                [{logon,fred}]]]},
      monitor={monSendLogon,[]},
      algorithm={mce_alg_safety,void}}).
\end{verbatim}


The result of the execution is given below:

\begin{verbatim}
Eshell V5.6.4  (abort with ^G)
1> run:logon().
Starting McErlang model checker environment...
Starting mce_alg_safety(void) algorithm on program
 scenario:start([[{logon,clara},{message,fred,"hola"},logoff],[{logon,fred}]])
with monitor monSendLogon([])


***Monitor failed***
monitor error:
  {somebody,logon}

Access result using mce:result()
To see the counterexample type "mce_erl_debugger:start(mce:result()). "

\end{verbatim}

We can now access the counter example:

\begin{verbatim}
2> mce_erl_debugger:start(mce:result()).
Starting debugger with a stack trace
Execution terminated with status:
  monitor failure.

...

stack(@13)> where().
13: 



12: process <n1,2> "registered as mess_client":
    receive {messenger,logged_on}
      in function await_result near line 169 of "messenger.erl"
        called by client near line 147 of "messenger.erl"
        with active data clara, server_node
        sent by node n1 in stack frame 11
    *** PROBE logon:clara ***



11: node server_node:
    received signal {signal,server_node,{message,{pid,n1,2},{messenger,logged_on}}} from node server_node
        sent by process server_node in stack frame 10

stack(@13)>
\end{verbatim}
The \texttt{where()} command displays the latest actions
of the process. In stack frame 12 we can see that
a message is sent to the messenger server,
and next the probe action occurs.


%%\bibliographystyle{abbrv}
%%\bibliography{}

\end{document}
